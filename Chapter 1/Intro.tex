
\chapter{Introduction}



\section{Motivation}
This thesis is centered on efficient algorithms for dealing with big data. In the modern world of large data sets, seemingly simple computational tasks can become infeasible when dealing with a massive data stream. 
In this thesis, we consider a similar problem known as \textit{Range Thresholding on Streams} which was first studied in 2016 \cite{Gan2016}. In the RTS problem, we consider a data stream of elements $\{e_t\}^{n}_{t=1}$ where each data element possess: 
\begin{itemize}
    \item A \textit{value} $v(e_t)\in\mathbb{R}^d$
    \item A \textit{weighting} $w(e_t)\in\mathbb{Z}$
\end{itemize}
We define a range-threshold query $q$ as a tuple $q = (R_q, \tau_q)$ where 
\begin{itemize}
    \item $R_q$ corresponds to some continuous (and possibly infinite) subinterval of $\mathbb{R}$. We refer to $R_q$ as the \textit{range} of the query. 
    \item $\tau_q \in\mathbb{N}$ corresponds to some threshold on the number of stream elements that enter the interval. We refer to $\tau_q$ as the \textit{threshold} of the query,
\end{itemize}
The goal of the RTS problem is to design an efficient data structure that reports the \textit{maturity} of a RTS query $q$. That is, if we denote by $S$ the subset of stream elements $\{e_t\}_{t=t^\prime}^{n}$ that have occurred \textit{after} $q$ was registered, and $v(e_t) \in R_{q}$, detect the precise moment that
\begin{align}
    \sum_{e_t \in S} w(e_t) \geq \tau_q
\end{align}
For a motivating example of where such a query could arise, we turn our attention to the financial markets. Consider a hypothetical stock trader who is interested in trading volumes around key price intervals. Such a trader may require alerts or notifications if trading volumes exceed some threshold, for example they may wish to register a query of the form: ``\textit{notify me when 100 units of \texttt{AAPL} trade between prices [300,305]}''. Such a notification would correspond the maturation of an RTS query over the stream of \texttt{AAPL}'s trading data, with query range $R_q = [300,305]$ and threshold $\tau_q = 100$. \\
\\
Similar to our earlier example of the \textit{Count Distinct} problem, managing RTS queries is deceptively difficult. Given a stream of $n$ elements and $m$ RTS queries registered on the stream, checking for condition $(1)$ via a brute-force approach requires a prohibitively large quadratic $O(nm)$ time and $\Omega(n+m)$ space. The goal of this thesis is to summarise the literature, and detail how to escape the quadratic run time. Next, we characterise when we can solve new type of RTS Query, where the interval is allowed to move over the lifetime of the algorithm. Finally, we propose and test optimisations to the state of the art solution to the RTS problem dependant on the data stream.

\section{Thesis Outline}

The remainder of this thesis is organised as follows; Chapter 2 is dedicated to a review and summary of the literature on RTS queries, as well as information on all necessary preliminaries to understand the state of the art solutions. Chapter 3 is dedicated to the original results of this thesis, from studying a new type of RTS query, to characterising when we can strengthen our theoretical bounds on runtime. Chapter 4 is dedicated to a performance study, where we implement and test our algorithms. Moreover, we provide an explanation of our methodology and testing environment for reproducability. Finally, we conclude our thesis is Chapter 5 which offers a summarised discussion of our results and offers relevant extensions to the direction of future research. As this thesis lies at the intersection of a traditional Computer Science thesis on Algorithms and Data Structures, we also provide an appendix with proofs of miscellaneous results for unfamiliar readers. 

