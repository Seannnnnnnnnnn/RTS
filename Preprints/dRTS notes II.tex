\documentclass{article}

% PACKAGES  
\usepackage[utf8]{inputenc}
\usepackage{amsbsy}
\usepackage{fixltx2e}
\usepackage{amsfonts}
\usepackage{natbib}
\usepackage{graphicx}
\usepackage{mathtools}
\usepackage{xcolor}
\usepackage{color} 
\usepackage{hyperref}
\usepackage{tcolorbox}
\usepackage{amsmath}
\usepackage{mathrsfs}
\usepackage{amssymb}
\usepackage{appendix}
\usepackage{amssymb}
\usepackage{amsthm}
\usepackage{fancyhdr}
\usepackage[T1]{fontenc}
\usepackage[colorinlistoftodos,prependcaption,textsize=tiny]{todonotes}
\usepackage[utf8]{inputenc}
\usepackage{lmodern}
\usepackage{minted}
\def\code#1{\texttt{#1}}
\usepackage{algorithm}
\usepackage{algpseudocode}
\usepackage[letterpaper,top=2cm,bottom=2cm,left=3cm,right=3cm,marginparwidth=2.0cm]{geometry}

\usepackage{pgfgantt}

% Theorem environments

\newtheorem{theorem}{Theorem}[section]
\newtheorem{corollary}{Corollary}[theorem]
\newtheorem{lemma}[theorem]{Lemma}
\newtheorem{definition}{Definition}
\newtheorem{example}{Example}
\newtheorem{remark}{Remark}


% Title page

\begin{document}
\hspace{95mm}\domark{Notes Prepared for Friday 3.10.2023}

\section*{Dynamic RTS Queries II}

We continue our discussion on the following type of Range Query: 

\begin{definition}[Dynamic RTS Query] A dynamic RTS (\textit{d-RTS}) query specifies a tuple $(R_{q(t)}, \tau_q)$ where $\tau_q \in \mathbb{N}$
    is a threshold, and $R_{q(t)} =[f_1(t), f_2(t)]$ is a variable subset of $R$, given by functions $f: t\rightarrow\mathbb{R}$ satisfying $f_1 < f_2$, and $t$ is the index of the data stream.
\end{definition}

Our problem formulation is to design an algorithm / data structure to efficiently process $m$ such queries. After some thought, we found we were able to find a method to solve one of the simpler forms of d-RTS queries, which we call \textit{equal-additive}:

\begin{definition}[Additive d-RTS Query]
    An Additive d-RTS query specifies a tuple $(R_{q(t)}, \tau_q)$ where $\tau_q\in \mathbb{N}$ is a threshold, and $R_{q(t)}$ is updated according to
    $$R_{q(t)} \leftarrow R_{q(t-1)} + \Delta_t$$
\end{definition}


\begin{definition}[Mutliplicative d-RTS Query]
    An Additive d-RTS query specifies a tuple $(R_{q(t)}, \tau_q)$ where $\tau_q\in \mathbb{N}$ is a threshold, and $R_{q(t)}$ is updated according to
    $$R_{q(t)} \leftarrow R_{q(t-1)} \cdot \Delta_t$$
\end{definition}







\end{document}
