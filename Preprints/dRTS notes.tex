\documentclass{article}

% PACKAGES  
\usepackage[utf8]{inputenc}
\usepackage{amsbsy}
\usepackage{fixltx2e}
\usepackage{amsfonts}
\usepackage{natbib}
\usepackage{graphicx}
\usepackage{mathtools}
\usepackage{xcolor}
\usepackage{color} 
\usepackage{hyperref}
\usepackage{tcolorbox}
\usepackage{amsmath}
\usepackage{mathrsfs}
\usepackage{amssymb}
\usepackage{appendix}
\usepackage{amssymb}
\usepackage{amsthm}
\usepackage{fancyhdr}
\usepackage[T1]{fontenc}
\usepackage[colorinlistoftodos,prependcaption,textsize=tiny]{todonotes}
\usepackage[utf8]{inputenc}
\usepackage{lmodern}
\usepackage{minted}
\def\code#1{\texttt{#1}}
\usepackage{algorithm}
\usepackage{algpseudocode}
\usepackage[letterpaper,top=2cm,bottom=2cm,left=3cm,right=3cm,marginparwidth=2.0cm]{geometry}

\usepackage{pgfgantt}

% Theorem environments

\newtheorem{theorem}{Theorem}[section]
\newtheorem{corollary}{Corollary}[theorem]
\newtheorem{lemma}[theorem]{Lemma}
\newtheorem{definition}{Definition}
\newtheorem{example}{Example}
\newtheorem{remark}{Remark}


% Title page

\begin{document}
\hspace{95mm}\domark{Notes Prepared for Friday 22.09.2023}

\section*{Dynamic RTS Queries}

\begin{definition}[Dynamic RTS Query] A dynamic RTS (\textit{d-RTS}) query specifies a tuple $(R_{q(t)}, \tau_q)$ where $\tau_q \in \mathbb{N}$
    is a threshold, and $R_{q(t)} =[f_1(t), f_2(t)]$ is a variable subset of $R$, given by functions $f: t\rightarrow\mathbb{R}$ satisfying $f_1 < f_2$, and $t$ is the index of the data stream.
\end{definition}

As it turns out, designing systems to manage \textit{d-RTS} queries can become rapidly complex depending on the behaviour of the functions $f$. As illustrated below, even in simple cases it appears difficult to escape a quadratic runtime of $O(nm)$ where $n$ is the size of the data stream, and $m$ the number of \textit{d-RTS} queries.

\subsubsection*{Case 1: ``Equal Step''}
Perhaps the most simple case of a \textit{d-RTS} query is the equal step case, where for each incoming stream element $e_t$; the end point of each query $R_q$ is shifted by a fixed step of $\Delta\in\mathbb{R}^d$. That is for each incoming stream element $e_t$ where $t = 1,\dots,$ the interval is updated according to: 
$$R_{q(t)} \leftarrow R_{q(t-1)} + \Delta$$
Which can be reformulated as 
\begin{align}
    R_{q(t)} \leftarrow R_{q(0)} + \Delta t
\end{align}
The expression in $(1)$ suggests a technique for solving the equal step case, which we call \textit{value shifting}. For the moment, we reinforce the \textit{one time registration constraint}; wherein all queries are registered at once, prior to the commencement of the data stream. We then apply the existing DT algorithm though for each stream element arriving at $t\geq1$, we update it's corresponding value according to 
\begin{align}
    v(e_t) \leftarrow v(e_t) - \Delta t
\end{align}
it is clear that $v(e_t) \in R_{q(t)} \iff v(e_t)-\Delta t \in R_{q(0)}$. Assuming that $v(e_t)\in\mathbb{R}^d$, performing the update step in $(2)$ requires $O(d)$ time, and thus results in $O(d\log^{d+1}m)$ time to process a single stream element, and thus $O(nd\log^{d+1}m)$ to process a stream of $n$ elements. \\
\\
As argued above, this adds an additional $O(d)$ factor to the runtime of the algorithms query processing component; which can be negligible for the typically small values $d$. The runtime can quickly explode however when we allow for additional queries to be registered over the lifetime of the algorithm. To see why, the \textit{shifting} rule for a \textit{d-RTS} query registered at time $t^\prime$ is
\begin{align}
    v(e_t) \leftarrow v(e_t) - \Delta(t - t^\prime)
\end{align}
Of course, on a root-to-leaf path, a stream element may cross many queries with various different registration times; that is, we could have $k$ different values of $t^\prime$ in $(3)$, and thus require $k$ different root-leaf path traces to ensure that all relevant nodes are visited. This results in $O(kd\log^{d+1}m)$ runtime to trace root-leaf paths which in the case where $k=m$ results in $O(ndm\log^{d+1}m)$ time to process $n$ stream elements. Finally, our analysis has so far neglected the need to consider the cost of logarithmic rebuilding, and additional space consumption that would be required to store counters of the requisite time stamps.\\
\\
Another option is to group queries based on the timestamp of their registration, and then run instances of the DT algorithm on each group. In the most pessimistic case, we can consider a user registering each query at differing timestamps; this results in $O(m)$ groups, and $O(m)$ \textit{shift calculations} in $(3)$. Tracing a root-leaf path now involves a very shallow tree, in particular of height 1, though ultimately results in $O(ndm)$ time to process $n$ stream elements.

\subsubsection*{Case 2: Logarithmic Groups}
The computational bottleneck that results in quadratic runtime in case 1 lies in $k$, the number of different timestamps $t$ where RTS queries are registered. As we have argued, in seems in the case where $k = O(m)$, solving $m$ RTS queries consumes at least $O(nm)$ time. \\
\\
In any case where $k = \Tilde{O}(1)$ however, we can solve the \textit{d-RTS} problem as described above in $\Tilde{O}(n+m)$ time using existing RTS algorithm problem. Another case to consider though, is if we can approximate the intervals of the $m$ RTS queries, so that they can be effectively grouped together. If we can sufficiently group the $m$ queries, particularly so that $m$ reduces to a logarithmic factor it appears then we are able to solve equal step d-RTS queries in $\Tilde{O}(n+m)$ time, as exemplified below.


\begin{example}[Binary Intervals, $d=1$] 
\end{example}
let $Q$ be the query set, and suppose instead we approximate each query such that the endpoints of its interval is shifted to the closest power of $2$ and let $R = \cup_{q\in Q}R_q$. To momentarily simplify the analysis, suppose that the leftmost endpoint of $R$ is 0, and the right most is $N$, this implies $h = \lfloor \log_2 N \rfloor$ powers of 2 lie in the interval $[0,N]$. The strategy of rounding to the nearest power of 2 results in the following number of possible endpoint pairs
\begin{align*}
    \frac{1}{2} {h\choose 2} &= \frac{h^2-h}{4} \\
    &= \frac{\lfloor\log_2 N\rfloor^2 - \lfloor\log N\rfloor}{4} \\
    &= O(\log N)
\end{align*}
For a quick explanation as to how we arrive at this, there are ${h\choose2}$ possible pairs of the $h$ endpoints, and only half of which are valid, with the left endpoint less than the right. \\
\\
The key observation here is that for a naive approach, this effectively reduces the number of queries whose interval we need to check from $\Omega(m)$ to $O(\log N)$, as now the endpoint of all intervals are now restricted to a power of 2. The challenge is now to manage query maturation as we process incoming stream elements. \\
\\
\textbf{One Time Registration Case}: Let's first simplify analysis and assume that all queries are preregistered. Recall that all queries now have their associated ranges set to one of $I_1, \dots, I_h$. Let $Q_j$ be the set of all thresholds of d-RTS queries with interval approximated to $I_j$, then $Q_j$ contains a set of associated d-RTS thresholds $\tau_1,\dots, \tau_i$. To solve the d-RTS problem, we maintain a sorted list of $\tau_1,\dots,\tau_i$ and process stream elements as follows.  \\
\\
For an incoming stream element $e_t$, we apply value shifting as per $(1)$, then check if $v(e_t) \in I_1, \dots, I_h$. For any interval satisfying $v(e_t)\in  I_j$, we identify the minimum threshold in $Q_j$ which we denote by $\tau_j^*$ and then updated according to: 
\begin{equation}
    \tau_j^* \leftarrow \tau_j^* - w(e_t)
\end{equation}
If this results in $\tau_j^* < 0$, then it is clear the associated RTS query has matured, and we remove $\tau^*_j$ from its sorted list in $O(1)$. Let $\tau^\prime_j$ denote the new minimum threshold in the list, to account for previous threshold updates, we now perform
\begin{equation}
    \tau_j^\prime \leftarrow \tau_J^\prime - \tau_j^*
\end{equation}
if this results in $\tau_j^\prime <0$, we repeat from calculation $(4)$, otherwise we are done. As $v(e_t)$ can only stab at most one of the $h$ intervals, we have now processed the stream element. \\
\\
\textit{Analysis}. The total run time of the above algorithm is as follows: preprocessing the intervals into approximate ranges $I_1, \dots, I_h$ requires $O(m)$\footnote{this is a little hand-wavey, but can be accomplished with a few passes over the $m$ queries}. We then must sort $Q_j$ for $j=1,\dots,h$. As the total number of thresholds across all $Q_j$ is $m$, constructing all the sorted lists requires $O(m\log m)$ time. To process an element $e_t$, the algorithm performs the shifting rule described in $(2)$, then checks if $v(e_t)$ stabs one of the $h$ intervals. In the case that $v(e_t)\in I_j$, we perform the update step in $(4)$, and if necessary $(5)$ with repetitions. \\
\\
It is important to observe, that in the one time registration case, calculation $(5)$ can only be performed a total of $m$ times over the life of the algorithm. For a stream of length $n$ therefore, the total run time for element processing is therefore $O(nh + m) = O(n\log N + m)$ combined with the cost to preprocess the queries, this brings the total runtime to $O(m\log m + n \log N) = \tilde{O}(n+m)$ \\
\\
\textbf{Dynamic Registration Case}: With a bit of thought, extending to this case is quite trivial. As before, approximate the queries associated with the intervals in $O(\log N)$ groups in $O(m\log m)$ time. When a new query $q = (R_{q(t)}, \tau_q)$ is registered, simply determine the new endpoints of $R_{q(t)}$, if this results in one of the already existing intervals $I_1, \dots, I_h$; simply insert $\tau_q$ into the sorted list associated with the interval in $O(m_{alive})$\footnote{assuming that the sorted lists are some array structure} time. \\
\\
\textit{Analysis.} This is argued exactly as before, though inserting a new element requires an additional $O(m_{alive})$ overhead to the runtime. 

\subsubsection*{Closing Remarks \& Next Steps}
In these notes, I give a definition of the dynamic RTS query, and explore the simplest type of d-RTS query, called the equal step case, wherein at each time step, endpoints are incremented according to some fixed value $\Delta \in\mathbb{R}$. Finally we propose a technique \textit{value shifting} for handling this case of d-RTS queries, and make the following arguments: 
\begin{enumerate}
    \item The DT algorithm studied so far can reduce to quadratic runtime of $O(ndm\log^{d+1}m)$ to process a stream of $n$ elements, which occurs when the $m$ queries are registered at different time stamps. In general for $k$ different time stamp registrations, runtime will be $O(ndk\log^{d+1}m)$, so it appears $k = \Tilde{O}(1)$ is required to remain subquadratic.
    \item Another alternative is to approximate the length of the intervals, so that a number of RTS queries can effectively be grouped together. We then show a relatively straightforward algorithm for handling $d=1$ queries which we argue takes $\Tilde{O}(n+m)$ runtime.
\end{enumerate}
Some thoughts for moving forward: 
\begin{enumerate}
    \item Do we certainly require quadratic runtime for the general case? or is there some trick that can be applied to circumvent this. 
    \item Can we easily extend the algorithm to support $d\geq1$ queries? 
    \item Is this interesting? The first of the two points above doesn't seem to bode well for the study of d-RTS queries, as quadratic runtime would suggest one to just use a naive approach. 
    \item If the answer to 3. is no, what else could we look at? Fortunately we have plenty of time to explore more problems! 
\end{enumerate}


\end{document}
