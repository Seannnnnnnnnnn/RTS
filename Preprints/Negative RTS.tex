\documentclass{article}

% PACKAGES  
\usepackage[utf8]{inputenc}
\usepackage{amsbsy}
\usepackage{fixltx2e}
\usepackage{amsfonts}
\usepackage{natbib}
\usepackage{graphicx}
\usepackage{mathtools}
\usepackage{xcolor}
\usepackage{color} 
\usepackage{hyperref}
\usepackage{tcolorbox}
\usepackage{amsmath}
\usepackage{mathrsfs}
\usepackage{amssymb}
\usepackage{appendix}
\usepackage{amssymb}
\usepackage{amsthm}
\usepackage{fancyhdr}
\usepackage[T1]{fontenc}
\usepackage[colorinlistoftodos,prependcaption]{todonotes}
\usepackage[utf8]{inputenc}
\usepackage{lmodern}
\usepackage{minted}
\def\code#1{\texttt{#1}}
\usepackage{algorithm}
\usepackage{algpseudocode}
\usepackage[letterpaper,top=2cm,bottom=2cm,left=3cm,right=3cm,marginparwidth=2.0cm]{geometry}

\usepackage[]{mdframed}


% draw a frame around given text
\newcommand{\framedtext}[1]{%
\par%
\noindent\fbox{%
    \parbox{\dimexpr\linewidth-2\fboxsep-2\fboxrule}{#1}%
}%
}

\usepackage{pgfgantt}

% Theorem environments

\newtheorem{theorem}{Theorem}[section]
\newtheorem{corollary}{Corollary}[theorem]
\newtheorem{lemma}[theorem]{Lemma}
\newtheorem{definition}{Definition}
\newtheorem{example}{Example}
\newtheorem{remark}{Remark}


% Title page

\begin{document}
\hspace{95mm}\domark{Notes Prepared for Friday 3.10.2023}

\section*{Negative-Weighted Data Streams}


Another extension to current research is to consider the case where stream weights can be negative integers. For a quick illustration as to why such an extension could be of practical importance, High Frequency Trading firms (HFTs) often consider RTS queries over the stream of \textit{limit order book insertions} \cite{HFTData}, which include \textit{matched} trades, \textit{limit} orders (orders to buy / sell at a specified future price) and \textit{cancellation orders}. Whilst matched and limit orders can be modelled in our current scenario, the cancellation orders would require an RTS algorithm to account for negative weights. With cancellation orders accounting for 96\% of order book transactions \citep{HFTData}, being able to model such scenarios is likely to be of interest to the HFT community. \\
\\
The previous algorithm used in \cite{Gan2016} is as follows: 

   \begin{mdframed}
First, if $\tau\leq6h$ tehn solve problem using the straight forward solution of $O(\tau) = O(h)$ words. 
    In the case that $\tau > 6h$ the coordinator shares every $s_i$ a slack value: 
    \begin{align}
        \lambda = \left\lfloor \frac{\tau}{2h}\right\rfloor
    \end{align}
    Then, $s_i$ intermittently sends a signal to the coordinator according to: 
    Let $\Bar{c}_i$ be the counter value at the time of last signal to $q$ (initally $\Bar{c}_i\leftarrow0$. Send a signal as soon as $c_i - \Bar{c}_i = \lambda$
    After receiving $h$ messages, the coordinator collects the precise counters $c_i$ from each participant, and rebegins the process on 
    \begin{align}
        \tau^\prime = \tau - \sum_{i=1}^{h}c_i
    \end{align}
\end{mdframed}

It is not mentioned in the original paper \cite{Cormode2011}, though this algorithm can also deal with negative counter updates. To see this, local counters can be updated to be negative, and the slack conditions will essentially be `readjusted' after $h$ messages have crossed when $(4)$ is conducted.  Moreover, Whenever $\tau\leq6h$ and a negative update results in $\tau > 6h$, we can proceed revert back to a round-based strategy. \\
\\
Thus, the algorithm remains correct in the presence of negative stream weights. Communication analysis is dificult though, and clearly depends on how \textit{negative} the stream weights are.

\newpage
\subsection*{Interval Counting on Streams}

Whilst looking around for prior to solutions to negative Distributed Tracking algorithms - I encountered and played around with the following paper \cite{10.1145/1142473.1142507}. With some analysis, I solved a different type of problem which could be described as \textit{Interval Counting on Streams} (ICS). We define this problem formally as follows: \\
\\
\textbf{The ICS problem setting}. We consider a data stream of $n$ elements $\{e_t\}_{t=1}^{n}$ where each element contains a weight $w(e_t)\in\mathbb{Z}$ and value $v(e_t)\in\mathbb{R}^d$. Moreover, we have a set of axis-parallel intervals $R_1,\dots,R_m\subset\mathbb{R}^d$. For each interval $R_i$ denote $S_i$ as the stream elements that have occurred after $R_i$ has been registered as an interval of interest, and for all $e\in S_i$ $v(e)\in R_i$. The goal of the algorithm is to maintain for each $i$: 
$$\sum_{e\in S_i}w(e_i)$$ \\
\\
\textbf{Naive Solution}. Similar to the RTS problem, escaping the quadratic trap is deceptively difficult. The Naive solution is to maintain a count $N_i$ for each interval $R_i$, and for each stream element, check if $v(e)\in R_i$ and then increment $N_i$. Given $m$ intervals and a stream of $n$ elements this is clearly $O(nm)$. \\
\\
\textbf{A Better Solution}. Looking at the ideas of \cite{10.1145/1142473.1142507} with the main DT algorithm, we can come up with a sub-quardatic solution.


Similar to other papers on distributed monitoring \cite{Cormode2011}, the paper models the scenario of a central coordinator $q$ which conducts pairwise 2-way communication with remote sites $s_1, \dots, s_h$. Each remote site $s_i$ observers a data stream $e_1, \dots, e_n$. Each remote site maintains a local estimate of it's stream
$$N_i(t) = \sum_{t^\prime < t} w(e_{t^\prime})$$
The cooridinator is interested in whenever the global count $N(t)$ over all streams where
$$N(t) = \sum_{i=1}^{h} N_i(t)$$
In particular, the paper provides a solution to this problem up to a $\delta$-\textit{efficient} threshold defined as follows 

\begin{definition}[$\delta$-efficient threshold count]
Given an error guarantee $\delta$ the $\delta$-efficient count, $\hat{N}$ satisfies the following: 
$$N(1-\delta) < \hat{N} < N$$
Where $N$ is the \textit{`true'} count 
\end{definition}

\subsection*{Algorithm Overview}
Similar to the distributed tracking algorithms we have seen before, the algorithm proposed in \cite{10.1145/1142473.1142507} sets local threhsolds at each site. Unlike the previous use of \textit{slack values}, at each site $S_i$ thresholds $t_{i,j}$ for $j=1,\dots$ are setso that the local count $\hat{N}_i(t)$ always satisfies
\begin{align}
    t_{i,f(i)} \leq \hat{N} < t_{i,f(i)+1}
\end{align}
If condition $(1)$ is violated, updates are sent to the coordinator and new thresholds are generated. Note that similiar to previous Distirbuted Tracking algorithms, the remote site ddoes note send any updates to the coordinator until the the actual local count lies outside these bounds. This means that the maximum error of $\hat{N}_i(t)$ is given by $t_{i,f(i)+1} - t_{i,f(i)}$. The paper considers two categories for setting such thresholds 
\begin{enumerate}
    \item \textbf{Static Thresholding}: each remote site is assigned a predetermined set of thresholds that do not change over the life of the algorithm . It simply tracks between which pair of thresholds its count currently lies, and informs the coordinator when this
changes
    \item \textbf{Adaptive Thresholding}: when old thresholds are violated, new thresholds at the remote sites are chosen by the central site according to the observed conditions, to dynamically reduce the communication overhead
\end{enumerate}
As noted by the authors While the adaptive thresholding methods can be expected to perform better than the static methods, the static methods are desirable when the capabilities of the remote site are limited due to the computational overhead that they add. For now, we only consider the three schemes proposed to set static local thresholds. In the communication cost analysis of each algorithm, we assume that updates are always positive. \\
\\
The paper provides a number of methods to set these local thresholds, one method we are particularly interested in is the following scheme: 

\subsubsection*{Proportional Threshold Scheme}

We can improve communication costs by setting a threshold that is proportional to the local counts. The following scheme is to use: 

$$t_j = (1+\delta)t_{j-1} \hspace{5mm} t_0 = 0, t_1 =1 $$
Recursively, this gives $t_{f(j)} = (1+\delta)^{f(j)-1}$ To see that this provides a $\delta$-efficient threshold count note that the global count error is bounded by 
\begin{align*}
    \hat{N} &\leq \sum_{i=1}^{h}N_i \\
            &\leq \sum_{i=1}^{h} t_{f(i)+1} - t_{f(i)}\\
            &=\sum_{i=1}^{h} \delta t_{f(i)} \\
            &\leq \delta \sum_{i=1}^{h}N_i = \delta N \hspace{20mm} \text{as $N_i\in [t_{f(i)}, t_{f(i)+1})$}
\end{align*}
To analyse the message cost, if $t_{f(i)}\leq N_i < t_{f(i)+1}$
then the total messages sent is $f(i)$. Since $t_{f(i)} = (1+\delta)^{f(i)+1}$ it follows that 
\begin{align*}
    f(i) &= 1 + \frac{\log (t_{f(i)})}{\log(1+\delta)} \\
         &\leq 1 + \frac{\log (N_i)}{\log(1+\delta)} \\
\end{align*}
The total messages sent from all sites therefore is 
\begin{align*}
    \sum_{i=1}^{h} f(i) &\leq \sum_{i=1}^{h}1 + \frac{\log (N_i)}{\log(1+\delta)} \\
    &= h + \frac{1}{\log(1+\delta)}\sum_{i=1}^{h}\log N_i \\
    &\leq h + h\frac{\log(\frac{N}{h})}{\log(1+\delta)}
\end{align*}
The final inequality comes noting that $\sum \log N_i = \log(\prod N_i)$ is maximised when $N_i = N/h$ for all $i$. The paper provides some further details on how to show this. Finally, we note that $\log^{-1}(1+\delta) = O(\frac{1}{\delta})$ and thus the total communication cost is $O(\frac{h}{\delta}\log\frac{N}{h})$


\subsubsection*{Negative Updates}
As noted in section 6 of the paper, the static protocols remain correct when negative updates are permitted. Instead of checking for upper bounds being exceeded, we must simply also check that the lower bound does in fact remain a lower bound. If either condition is violated, local error bounds are recomputed and the coordinator is informed. If of interest, the paper also proposes ways to extend the adaptive threshold schemes so that negative updates are also allowed. \\
\\
Of course, the bounds on communication no longer quite hold, as there could be additional messages. Other work has explored the complexity of distributed monitoring when negative updates are allowed \cite{DTwithoutmonotonicity}, though the bounds obtained require some assumptions on the distribution of the data stream, which we do not make in our works.


\subsection*{Extending the DT algorithm}

We now want to leverage this new algorithm to modify the existing RTS solution from \cite{Gan2016} to support registering interval-counting queries on a data stream, where each query $q_i$ corresponds to an interval of interest $R_i$.\\
\\
The algorithm proceeds as before, given $m$ interval queries, we register them as RTS queries of the form $q_i = (R_i, +\infty)$. We construct the end point tree $T$ on the endpoints of $R_1,\dots,R_m$. As before, each internal node in $T$ corresponds to a jurisdiction interval, and we define the \textit{canonical node set} $Q_i$ of each query $q_i$ as the smallest collection of vertices whose union of jurisdiction intervals is $R_i$. Finally, each node in $v\in V(T)$ has a corresponding counter $c_v \leftarrow 0$\\
\\
To process a stream element $e_t$, we trace root-leaf paths based on $v(e_t)$, updating counters located at vertices along the path. As before, we need to consider the \textit{slack inspection} costs and when messages should be updated as part of the slack inspection. In \cite{Gan2016} this is done by maintaining min-heaps at each internal node by the key:
\begin{align*}
    \sigma_q &= \lambda_q + \Bar{c}_q(u) \\
              &= \left\lfloor \frac{\tau_q}{2|Q_q|}\right\rfloor + \Bar{c}_{q}(u)
\end{align*}
Where $\lambda_q$ above is the slack value from the previous distributed tracking algorithm. In our new algorithm, a participant $s_i$ maintains local thresholds $t_{f(i)}, t_{f(i)+1}$ and messages the coordinator whenever it's local counter $\hat{N}\notin [t_{f(i)}, t_{f(i)+1}]$. Thus, if we denote our local counter by $\hat{N}$, we can borrow the previous min-heap strategy with keys: 
\begin{align*}
    \sigma_q &= \min(t_{f(i)+1}-\hat{N}, \hat{N}-t_{f(i)}) + \Bar{N}_q 
\end{align*}
Where $\Bar{N}_q$ was the value of the counter at the previous communication.


\subsubsection*{Analysis}
The analysis of the algorithm is as before. To initially simplify the analysis, we restrict ourselves to the one-time registration case and the dimensionality of the data stream values to $d=1$. As before, constructing the endpoint tree requires $O(m\log m)$. To process an a data stream of $O(n)$ elements, we trace a root-leaf path for each element and then perform slack inspections and message costs for the DT instances. The path tracing clearly requires $O(n\log m)$. The messaging cost requires $O(\frac{h_q}{\delta}\log \frac{N}{h_q})$. In the RTS context this is therefore 
\begin{align}
    O\left(\frac{h_q}{\delta}\log \frac{N_i}{h_q}\right) = O(\log m \log N_{i})
\end{align}
Provided the above is correct - analysis follows identically to \cite{Gan2016}. This results in a runtime of $O(n\log m + m\log^2 m\log N_{max})$

\subsubsection*{TODO}
if interesting, flesh out the analysis for the more general case where we want to count stream elements over axis-parrallel ranges. Moreover, formally define the data structures methods for reportings counts.

\newpage
\bibliographystyle{unsrt}%Used BibTeX style is unsrt
\bibliography{sample}

\end{document}
