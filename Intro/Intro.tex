\section{Introduction \& Motivation}
Within this thesis we explore current state of the art solutions to the stream-computing / computational geometry problem of \textit{Range Thresholding on Streams} (RTS) \cite{Gan} which was first studied in 2016. In the RTS problem, we consider a data stream of elements $\{e_t\}$ where each data element possess: 
\begin{itemize}
    \item A \textit{value} $v(e_t)\in\mathbb{R}^d$
    \item A \textit{weighting} $w(e_t)\in\mathbb{N}$
\end{itemize}
On the data stream we define a RTS query $q$ as a touple $q = (R_q, \tau_q)$ where 
\begin{itemize}
    \item $R_q$ corresponds to some continuous (and possibly infinite) subinterval of $\mathbb{R}$. We refer to $R_q$ as the \textit{range} of the query. 
    \item $\tau_q \in\mathbb{N}$ corresponds to some threshold on the number of stream elements that enter the interval. 
\end{itemize}
The goal of the RTS problem is to design an efficient data structure that reports the \textit{maturity} of a RTS query $q$, where we define maturity to be the smallest such $t$ such that 
$$\sum_{i = 1}^{t} v(e_t) = \tau_q$$
